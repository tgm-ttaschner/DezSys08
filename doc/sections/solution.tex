%!TEX root=../document.tex

\section{Ergebnisse}
\label{sec:Ergebnisse}


\subsection{Nutzung von GPUs in normalen Desktop-Anwendungen}
\label{sec:Nutzung von GPUs in normalen Desktop-Anwendungen}
Im privaten Bereich gibt es praktisch keine Anwendung für GPGPU. Ausnahmen sind Programme für die Videobearbeitung. Typische Alltagssoftware lässt sich kaum parallelisieren. Im Bereich der Wissenschaft und Technik sieht es dann schon wieder anders aus. Hier machen mehrere GPGPU-taugliche Grafikkarten einen normalen PC zu einem Supercomputer.
Voraussetzung dafür ist ein hochleistungsfähiger Grafikprozessor, der bis zu 800 Kerne hat und parallel arbeitet. Damit ein Grafikprozessor überhaupt sinnvoll genutzt werden kann, muss die Anwendung eine sehr hohe Zahl von parallel ausführbaren Berechnungen enthalten. Wenn man die Rechenwerke eines Grafikprozessors kontinuierlich auslasten will, dann sollte man mindestens eine Millionen gleichartiger Berechnungen liefern können, sonst hat man gegenüber einer normalen CPU keine Vorteile. Die GPU-Rechenwerke müssen mit einem kontinuierlichen Datenstrom versorgt werden. Aus diesem Grund bezeichnet man sie auch als Stream-Prozessoren. \cite{elektronik_gpgpu}

\subsubsection{Anwendungen}
\label{sec:Anwendungen}
\begin{itemize}
\item Physikalisch Simulationen (Strömung, Gravitation, Temperatur, Crash-Tests)
\item Komplexe Klimamodelle (Wettervorhersage)
\item Datenanalysen und Finanzmathematik
\item Verarbeitung von akustischen und elektrischen Signalen
\item CT- und Ultraschall-Bildrekonstruktion
\item Modellieren von Molekühlstrukturen
\item Kryptografie
\item Neuronale Netze
\item Videotranskodierung
\end{itemize}

\subsection{Vorteile der GPU bei rechenintensiven Anwendungen}
\label{sec:Vorteile der GPU bei rechenintensiven Anwendungen}
